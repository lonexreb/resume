%%%%%%%%%%%%%%%%%
% This is an example CV created using altacv.cls (v1.1, 21 November 2016) written by
% LianTze Lim (liantze@gmail.com), based on the 
% Cv created by BusinessInsider at http://www.businessinsider.my/a-sample-resume-for-marissa-mayer-2016-7/?r=US&IR=T
% 
%% It may be distributed and/or modified under the
%% conditions of the LaTeX Project Public License, either version 1.3
%% of this license or (at your option) any later version.
%% The latest version of this license is in
%%    http://www.latex-project.org/lppl.txt
%% and version 1.3 or later is part of all distributions of LaTeX
%% version 2003/12/01 or later.
%%%%%%%%%%%%%%%%

%% If you want to use \orcid or the
%% academicons icons, add "academicons"
%% to the \documentclass options. 
%% Then compile with XeLaTeX or LuaLaTeX.
% \documentclass[10pt,a4paper,academicons]{altacv}
\documentclass[10pt,a4paper]{altacv}

%% AltaCV uses the fontawesome and academicon fonts
%% and packages. 
%% See texdoc.net/pkg/fontawecome and http://texdoc.net/pkg/academicons for full list of symbols.
%% When using the "academicons" option,
%% Compile with LuaLaTeX for best results. If you
%% want to use XeLaTeX, you may need to install
%% Academicons.ttf in your operating system's font %% folder.


% Change the page layout if you need to
\geometry{left=1cm,right=9cm,marginparwidth=6.8cm,marginparsep=1.2cm,top=1cm,bottom=1cm}

% Change the font if you want to.

% If using pdflatex:
\usepackage[utf8]{inputenc}
\usepackage[T1]{fontenc}
\usepackage[default]{lato}

% If using xelatex or lualatex:
% \setmainfont{Lato}

% Change the colours if you want to
\definecolor{VividPurple}{HTML}{3E0097}
\definecolor{SlateGrey}{HTML}{2E2E2E}
\definecolor{LightGrey}{HTML}{666666}
\colorlet{heading}{VividPurple}
\colorlet{accent}{VividPurple}
\colorlet{emphasis}{SlateGrey}
\colorlet{body}{LightGrey}

% Change the bullets for itemize and rating marker
% for \cvskill if you want to
\renewcommand{\itemmarker}{{\small\textbullet}}
\renewcommand{\ratingmarker}{\faCircle}

%% sample.bib contains your publications
\addbibresource{sample.bib}

\begin{document}
\name{Shubhankar Tripathy}
%\tagline{Senior Software Engineer - iOS}
\personalinfo{%
  % Not all of these are required!
  % You can add your own with \printinfo{symbol}{detail}
  \email{stripathy@umass.edu}
%   \phone{+49-173-6895039}
  %\mailaddress{Address: Maruthi Nagar, Banglore}
  \location{Amherst, MA}
  \linkedin{linkedin.com/in/shubhankar-tripathy}
  \phone{+1 (413) 612 7524}   \github{github.com/lonexreb} 
  %\printinfo{\faMedium}{medium.com/@abhimuralidharan}
 % \printinfo{\faStackOverflow}{https://goo.gl/zBv3hW}
  
  % I'm just making this up though.
%   \orcid{orcid.org/0000-0000-0000-0000} % Obviously making this up too. If you want to use this field (and also other academicons symbols), add "academicons" option to \documentclass{altacv}
}

%% Make the header extend all the way to the right, if you want. Extend the right margin by 8cm (=6.8cm marginparwidth + 1.2cm marginparsep)
\begin{adjustwidth}{}{-8cm}
\makecvheader
\end{adjustwidth}

%% Provide the file name containing the sidebar contents as an optional parameter to \cvsection.
%% You can always just use \marginpar{...} if you do
%% not need to align the top of the contents to any
%% \cvsection title in the "main" bar.
\cvsection[page1sidebar]{Experience}
\cvevent{Tech and Product Management Intern}{DegreeSight}{July, 2022 -- Present}{Austin, TX}
\begin{itemize}
\item Implement innovative strategies to promote sales and adoption of platform, which allows users to seamlessly transfer between institutions of higher education.
\item Maintain efficient product management strategy using tools including Mindmapping, Confluence, JIRA and complex data analytics.
% \item Hands on experience on iOS app called EventSecurity Committee
\end{itemize}
\divider

\cvevent{Undergraduate Researcher}{CICS, UMass Amherst}{Dec '20 -- Jan '21}{Amherst, MA}
\begin{itemize}
\item Programmed adaptive sampling method using scikit-learn library in Python. 
\item Developed predictive mapping algorithm for 2D scaler fields utilizing Gaussian process.
\item Presented final research to students, faculty and staff of CICS.
\item Link to the presentation \url{https://bit.ly/URV-W21}.
% \item End to end UI development and implementation for  multiple screens .
% \item Created custom UI controls and performance optimization.

% \item Mentoring and review code for a trainee.

% \item Worked on full application development life cycle.
% \item Handling team.
\end{itemize}
\divider

\cvevent{Data Analyst Intern}{Shyam Metallics}{Nov '18 -- May '19}{Sambalpur, Odisha, India}
\begin{itemize}
\item Built and improved projections model with better accuracy and forecasted the volume of business using Pandas, NumPy, and MatPloLib libraries of Python. 
\item Coded in Python to build algorithmic programs to solve pipe and cisterns mathematical problems for Utility Department.
\item•	Reduced revenue loss by 15.5\% by optimizing datasets of raw material intake, using SQL queries.
% \item End to end UI development and implementation for  multiple screens .
% \item Created custom UI controls and performance optimization.

% \item Mentoring and review code for a trainee.

% \item Worked on full application development life cycle.
% \item Handling team.
\end{itemize}


% \divider

% \cvevent{Product Engineer}{Google}{23 June 1999 -- 2001}{Palo Alto, CA}

% \begin{itemize}
% \item Joined the company as employe \#20 and female employee \#1
% \item Developed targeted advertisement in order to use user's search queries and show them related ads
% \end{itemize}

\cvsection[page1sidebar]{Project}
\cvevent{FoodMood}{Winner @TreeHacks - Hackthon  \url {https://tinyurl.com/foodmood-treehacks}}{Feb '22}{Stanford '22}
\begin{itemize}
\item	Built a web system that can intelligently predict the user’s mood based on the food they have eaten. 
\item Used computer vision tools including Bubble Clarai and InterSystems FIHR API to build web system. 
\item Classified data in CSV files and built the Computer Vision backend, Generative Adversarial Network (GAN).
\item Completed project through testing phase in under 36 hours.

% \item Hands on experience on iOS app called EventSecurity Committee
\end{itemize}
\divider


\cvevent{Course Projects - Accomplished during coursework}{ *Request private access**Academic Honesty Policy*}{Fall '20 - Present}{Amherst, MA}


\begin{itemize}
\item \url{https://github.com/lonexreb/Systems-Programming}
\item \url{https://github.com/lonexreb/DataStructureAndAlgo_CS187}
% \item Hands on experience on iOS app called EventSecurity Committee
\end{itemize}
\divider

\clearpage


\end{document}
